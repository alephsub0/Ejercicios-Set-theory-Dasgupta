\documentclass[a4,10pt]{aleph-notas}

% -- Paquetes adicionales
\usepackage{enumitem}
\usepackage{aleph-comandos}

% -- Datos
\institucion{Proyecto Alephsub0}
\asignatura{Complejos simpliciales para nubes de datos}
\tema{Vietoris-Rips vs. Čech}
\autor{Andrés Merino}
\fecha{Noviembre 2025}

\fuente{montserrat}
\logouno[4.5cm]{Logos/LogoAlephsub0-02}
\definecolor{colordef}{cmyk}{0.81,0.62,0.00,0.22}

% -- Comandos adicionales
\newtheorem*{prob}{Problema}
    \tcolorboxenvironment{prob}{%
        color=colordef,recuadrost,colback=colordef!7,drop fuzzy shadow
    }


\begin{document}

\encabezado

\noindent Consideremos las siguientes definiciones de complejos simpliciales que se definen a partir de una nube de datos (punto en un espacio métrico).


\begin{defi}[Vietoris–Rips complex]
Sea \( X \) un subconjunto de un espacio métrico y \( \alpha > 0 \).  
El \emph{complejo de Vietoris–Rips} \( \mathrm{VR}_\alpha(X) \) es el conjunto de símplices 
\([x_0, \dots, x_k]\) tales que, para todo $(i,j)$,
\[
d(x_i, x_j) \leq \alpha.
\]
\end{defi}

\begin{defi}[Čech complex]
Sea \( X \) un subconjunto de un espacio métrico y \( \alpha > 0 \).  
El \emph{complejo de Čech} \( \mathrm{Cech}_\alpha(X) \) es el conjunto de símplices 
\([x_0, \dots, x_k]\) tales que las bolas cerradas
\[
\bigcap_{i=0}^k B(x_i, \alpha) \neq \emptyset.
\]
\end{defi}

Existe la siguiente relación entre ellos.

\begin{prob}
Sea \( X \) un subconjunto de un espacio métrico. Para todo \( \alpha > 0 \) se tiene la siguiente cadena de inclusiones:
\[
\mathrm{VR}_\alpha(X) \subseteq \mathrm{Cech}_\alpha(X) \subseteq \mathrm{VR}_{2\alpha}(X).
\]
\end{prob}

\begin{proof}
Sea \([x_0, \dots, x_k] \in \mathrm{VR}_\alpha(X)\).  
Por definición, se cumple que, para todo $(i,j)$,
\[
d(x_i, x_j) \leq \alpha.
\]
Consideremos el punto \( x_0 \).  
Dado que \( d(x_0, x_i) \leq \alpha \) para todo \( i \), se tiene que 
\( x_0 \in B(x_i, \alpha) \) para todo \( i \).  
Por lo tanto,
\[
x_0 \in \bigcap_{i=0}^k B(x_i, \alpha),
\]
y la intersección no es vacía.  
De esto se deduce que 
\([x_0, \dots, x_k] \in \mathrm{Cech}_\alpha(X)\),  
es decir,
\[
\mathrm{VR}_\alpha(X) \subseteq \mathrm{Cech}_\alpha(X).
\]

\vspace{1em}
Para la inclusión contraria, sea \([x_0, \dots, x_k] \in \mathrm{Cech}_\alpha(X)\).  
Entonces,
\[
\bigcap_{i=0}^k B(x_i, \alpha) \neq \emptyset.
\]
Es decir, existe un punto \( z \in X \) tal que \( d(x_i, z) \leq \alpha \) para todo \( i \).  
Tenemos que, para cada par \((i,j)\),
\[
d(x_i, x_j) \leq d(x_i, z) + d(z, x_j) \leq \alpha + \alpha = 2\alpha.
\]
Por lo tanto, \([x_0, \dots, x_k] \in \mathrm{VR}_{2\alpha}(X)\), de donde, 
\[
\mathrm{Cech}_\alpha(X) \subseteq \mathrm{VR}_{2\alpha}(X).
\]
\end{proof}


\end{document}