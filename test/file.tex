\documentclass[a4,10pt]{aleph-notas}

% -- Paquetes adicionales
\usepackage{enumitem}
\usepackage{aleph-comandos}

% -- Datos del libro
\institucion{Proyecto Alephsub0}
% \carrera{}
\asignatura{Teoría de la Medida}
\tema{Ejercicio de prueba}
\autor{Andrés Merino}
% \fecha{Abril 2024}
\fuente{montserrat}


%% --> Logos de las guias
\logouno[4.5cm]{Logos/LogoAlephsub0-02}
\definecolor{colordef}{cmyk}{0.81,0.62,0.00,0.22}


% -- Comandos adicionales
\newcommand{\A}{\mathcal{A}}
\newcommand{\B}{\mathcal{B}}
\renewcommand{\C}{\mathcal{C}}

\begin{document}

\encabezado

%%%%%%%%%%%%%%%%%%%%%%%%%%%%%%%%%%%%%%%%
%%%%%%%%%%%%%%%%%%%%%%%%%%%%%%%%%%%%%%%%
%%%%%%%%%%%%%%%%%%%%%%%%%%%%%%%%%%%%%%%%

\begin{ejer}
    Enunciado del ejercicio. Sea $\A \in \B$. $\sen(x)$ $\dlim_{x\to 0} f(x)$.
\end{ejer}


Aquí vamos a probar algunas cosas. Los símbolos como 
\[
    \funcion{f}{\R}{\R}{x}{x^2}
\]
Por otro lado, símbolos $a<b$ o $a>b$, $\dfrac{a}{b}$. Un \texttt{align}:
\begin{align*}
    a &= b\\
    c &= d
\end{align*}
Y probemos una referencia a la ecuación \eqref{eq:1}.

También probemos texto en \text{negritas} y \textit{cursivas}. Y una lista:
\begin{itemize}
    \item Primer elemento.
    \item Segundo elemento.
    \item Tercer elemento.
\end{itemize}
Y una tabla:
\begin{center}
    \begin{tabular}[h]{|c|c|}
        \hline
        $a$ & $b$\\
        \hline
        $c$ & $d$\\
        \hline
    \end{tabular}
\end{center}



\begin{ejer}
    Enunciado del ejercicio. Sea $\A \in \B$. $\sen(x)$ $\dlim_{x\to 0} f(x)$.
\end{ejer}

\begin{proof}
    Redacción de la demostración.\qedhere
\end{proof}

Esta es la ecuación:
\begin{equation}
    a = b.
    \label{eq:1}
\end{equation}

\end{document}