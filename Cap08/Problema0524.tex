\documentclass[a4,10pt]{aleph-notas}

% -- Paquetes adicionales
\usepackage{enumitem}
\usepackage{aleph-comandos}

% -- Datos
\institucion{Proyecto Alephsub0}
\asignatura{Teoría de conjuntos}
\tema{Problema 524 del libro de Abhijit Dasgupta}
\autor{Andrés Merino}
\fecha{Julio 2025}

\fuente{montserrat}
\logouno[4.5cm]{Logos/LogoAlephsub0-02}
\definecolor{colordef}{cmyk}{0.81,0.62,0.00,0.22}

% -- Comandos adicionales
\newtheorem*{prob}{Problema}
    \tcolorboxenvironment{prob}{%
        color=colordef,recuadrost,colback=colordef!7,drop fuzzy shadow
    }
\usepackage{pgfplots}
\pgfplotsset{compat=1.18}

\begin{document}

\encabezado

\noindent
El siguiente problema corresponde al Problema 524 del libro Set theory de Abhijit Dasgupta.

\begin{prob}
    Da un ejemplo de dos órdenes no isomorfos, cada uno de los cuales sea isomorfo a un suborden del otro.
\end{prob}

\begin{proof}
    Consideremos los conjuntos ordenados
    \[
        E = [0,1] \quad \text{y} \quad F = (0,1].
    \]
    No son isomorfos, pues $E$ tiene un elemento mínimo ($0$) y $F$ no tiene elemento mínimo.

    Por otro lado, definamos las aplicaciones:
    \[
        \funcion{f_1}{E}{F}{x}{\frac{x+1}{2},}
        \qquad\text{y}\qquad
        \funcion{f_2}{F}{E}{x}{x.}
    \]
    Para $f_1$, observemos que:
    \begin{itemize}
        \item Para $x, y$ en $E$, se tiene que
        \[
            x < y
            \qDimp
            \frac{x+1}{2} < \frac{y+1}{2}
            \qDimp
            f_1(x) < f_2(x).
        \]
        \item Por el punto anterior, $f_1$ es inyectiva.
        \item La imagen de $f_1$ es $f_1(E) = \left[\frac{1}{2}, 1\right] \subseteq F$.
    \end{itemize}
    Por lo tanto, $f_1$ es un isomorfismo de órdenes entre $E$ y el suborden $f_1(E)$ de $F$.

    Análogamente, para $f_2$:
    \begin{itemize}
        \item $f_2$ es la inclusión de $F$ en $E$, por lo que, para $x,y\in E$,
        \[
            x < y
            \qDimp
            f_2(x) < f_2(x).
        \]
        \item Por el punto anterior, $f_2$ es inyectiva.
        \item La imagen de $f_2$ es $F$ mismo.
    \end{itemize}
    Por lo tanto, $f_2$ es un isomorfismo de órdenes entre $F$ y el suborden $F \subseteq E$.

    Así, aunque $E$ y $F$ no son isomorfos, cada uno es isomorfo a un suborden del otro.
\end{proof}


\end{document}