\documentclass[a4,10pt]{aleph-notas}

% -- Paquetes adicionales
\usepackage{enumitem}
\usepackage{aleph-comandos}

% -- Datos
\institucion{Proyecto Alephsub0}
\asignatura{Teoría de conjuntos}
\tema{Problema 521-1 del libro de Abhijit Dasgupta}
\autor{Andrés Merino}
\fecha{Julio 2025}

\fuente{montserrat}
\logouno[4.5cm]{Logos/LogoAlephsub0-02}
\definecolor{colordef}{cmyk}{0.81,0.62,0.00,0.22}

% -- Comandos adicionales
\newtheorem*{prob}{Problema}
    \tcolorboxenvironment{prob}{%
        color=colordef,recuadrost,colback=colordef!7,drop fuzzy shadow
    }
\usepackage{pgfplots}
\pgfplotsset{compat=1.18}

\begin{document}

\encabezado

\noindent
El siguiente problema corresponde al Problema 521-1 del libro Set theory de Abhijit Dasgupta. Primero, presentemos algunas definiciones:

\begin{defi}
    Toda cortadura \( (L, U) \) en un orden \( E \) es exactamente uno de los siguientes tipos:
    
    \begin{enumerate}
        \item \textbf{Salto:} \( L \) tiene un máximo y \( U \) tiene un mínimo.
        \item \textbf{Corte por punto límite superior:} \( L \) no tiene máximo, pero \( U \) tiene un mínimo. Ese mínimo es un punto límite superior de \( L \).
        \item \textbf{Corte por punto límite inferior:} \( U \) no tiene mínimo, pero \( L \) tiene un máximo. Ese máximo es un punto límite inferior de \( U \).
        \item \textbf{Hueco:} Ni \( L \) tiene un máximo, ni \( U \) tiene un mínimo.
    \end{enumerate}
\end{defi}

\begin{defi}
    Un orden \( E \) es \textbf{Dedekind completo} si no tiene huecos, es decir, si toda cortadura en \( E \) no es un hueco.
\end{defi}


\begin{defi}
    Un orden \( E \) es \textbf{totalmente discreto} si toda cortadura en \( E \) es un salto.
\end{defi}


\begin{prob}
    Un orden \( E \) es totalmente discreto si y sólo si es Dedekind completo y no tiene puntos límite.
\end{prob}

\begin{proof}
Supongamos que \( E \) es totalmente discreto. Es decir, toda cortadura en \( E \) es un salto.

\begin{itemize}
    \item \textit{Completitud:} Sea \( (L, U) \) una cortadura en \( E \). Como todo corte es un salto, entonces ningún corte puede ser un hueco. Por lo tanto, \( E \) es Dedekind completo.

    \item \textit{Ausencia de puntos límite:} Supongamos, para obtener una contradicción, que \( E \) tiene un punto límite. 
    Sin pérdida de generalidad, supongamos que \( a \in E \) es un punto límite superior. 
    Entonces, por definición, $a$ no es el primer elemento de $E$ y para todo \( x < a \), existe \( p \in A \) tal que \( x < p < a \).

    Consideremos:
    \[
    L = \{x \in E : x < a\}, \quad U = \{x \in E : x \ge a\}.
    \]
    Como $a$ no es el primer elemento de $E$, $L\neq \emptyset$, de donde $(L,U)$ es una cortadura. Por ser \( E \) totalmente discreto, debe ser un salto. 
    Por lo tanto, \( L \) tiene un máximo, sea \( b = \max(L) \).

    Sin embargo, por la definición de punto límite superior, debe existir \( p \in A \) tal que \( b < p < a \), lo cual contradice que \( b \) sea el máximo de \( L \). 
    Por lo tanto, nuestra suposición era falsa, y \( E \) no tiene puntos límite.
\end{itemize}

Recíprocamente, supongamos que \( E \) es Dedekind completo y no tiene puntos límite. 
Sea \( (L, U) \) una cortadura en \( E \).

Como \( E \) es completo, la cortadura no puede ser un hueco. 
Como no tiene puntos límite, la cortadura no puede ser ni por punto límite superior ni por punto límite inferior. 
Entonces, la única posibilidad es que la cortadura sea un salto. Por lo tanto, toda cortadura en \( E \) es un salto, es decir, \( E \) es totalmente discreto. 
\end{proof}

\end{document}