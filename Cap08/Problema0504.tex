\documentclass[a4,10pt]{aleph-notas}

% -- Paquetes adicionales
\usepackage{enumitem}
\usepackage{aleph-comandos}

% -- Datos
\institucion{Proyecto Alephsub0}
\asignatura{Teoría de conjuntos}
\tema{Problema 504 del libro de Abhijit Dasgupta}
\autor{Andrés Merino}
\fecha{Mayo 2025}

\fuente{montserrat}
\logouno[4.5cm]{Logos/LogoAlephsub0-02}
\definecolor{colordef}{cmyk}{0.81,0.62,0.00,0.22}

% -- Comandos adicionales
\newtheorem*{prob}{Problema}
    \tcolorboxenvironment{prob}{%
        color=colordef,recuadrost,colback=colordef!7,drop fuzzy shadow
    }


\begin{document}

\encabezado

\noindent
El siguiente problema corresponde al Problema 504 del libro Set theory de Abhijit Dasgupta. Primero, presentemos algunas definiciones:

\begin{defi}
Sean $X$ un conjunto ordenado, $a \in X$ y  $A \subseteq X$.
Decimos que $a$ es un \textit{punto de acumulación superior de $A$ en $X$} si $a$ no es el primer elemento de $X$, y para todo $x < a$ existe algún $p \in A$ tal que $x < p < a$. 
\end{defi}

\begin{defi}
Un conjunto ordenado no trivial $X$ se dice \textit{denso en orden} si para todos $x, y \in X$ con $x < y$, existe $z \in X$ tal que $x < z < y$, es decir, si $X$ no contiene elementos consecutivos.
\end{defi}

\begin{prob}
    Sea $X$ un conjunto ordenado no trivial. Demuestre que todo elemento de $X$ es un punto de acumulación superior si y solo si $X$ es denso en orden y no tiene primer elemento.
\end{prob}

\begin{proof}
Supongamos que todo elemento de $X$ es un punto de acumulación superior.

\begin{itemize}
    \item Como todo elemento de $X$ es un punto de acumulación superior, entonces ninguno puede ser el primer elemento. Por lo tanto, $X$ no tiene primer elemento.

    \item Ahora, sean $x, y \in X$ tales que $x < y$. Como $y$ es un punto de acumulación superior, se tiene que para todo $x < y$, existe $z \in X$ tal que $x < z < y$. De donde se concluye que $X$ es denso en orden.
\end{itemize}

Recíprocamente, supongamos que $X$ es denso en orden y que no tiene primer elemento. Sea $x \in X$, vamos a demostrar que $x$ es un punto de acumulación superior.

Primero, como $X$ no tiene primer elemento, se cumple que $x$ no es el primer elemento de $X$. Además, para todo $y < x$ con $y \in X$, como $X$ es denso en orden, existe $z \in X$ tal que $y < z < x$. Por lo tanto, existe un elemento entre $y$ y $x$ que pertenece a $X$, lo que implica que $x$ es un punto de acumulación superior.
\end{proof}

\end{document}