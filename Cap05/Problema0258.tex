\documentclass[a4,10pt]{aleph-notas}

% -- Paquetes adicionales
\usepackage{enumitem}
\usepackage{aleph-comandos}

% -- Datos
\institucion{Proyecto Alephsub0}
\asignatura{Teoría de conjuntos}
\tema{Problema 258 del libro de Abhijit Dasgupta}
\autor{Andrés Merino}
\fecha{Octubre 2024}

\fuente{montserrat}
\logouno[4.5cm]{Logos/LogoAlephsub0-02}
\definecolor{colordef}{cmyk}{0.81,0.62,0.00,0.22}

% -- Comandos adicionales
\newtheorem*{prob}{Problema}
    \tcolorboxenvironment{prob}{%
        color=colordef,recuadrost,colback=colordef!7,drop fuzzy shadow
    }


\begin{document}

\encabezado

\noindent
El siguiente problema corresponde al Problema 258 del libro Set theory de Abhijit Dasgupta. Primero, presentemos algunas definiciones:

\begin{defi}
    Dos conjuntos $A$ y $B$ se llaman \textbf{equinumerosos}, denotado como $A \sim B$, si existe una función biyectiva de $A$ en $B$.
\end{defi}

\begin{defi}
    Un conjunto $A$ se dice \textbf{Dedekind-infinito} si $A \sim B$ para algún subconjunto propio $B \subseteq A$, es decir, si existe una función $\func{f}{A}{A}$ que es inyectiva pero no sobreyectiva. Un conjunto se denominará \textbf{Dedekind-finito} si no es Dedekind-infinito.
\end{defi}

\begin{prob}
    Si $A\sim B$ y $A$ es Dedekind-finito, entonces $B$ también lo es.
\end{prob}

\begin{proof}
    Sean \( A \) y \( B \) tales que \( A \sim B \), es decir, existe una función biyectiva \( \varphi: A \to B \). Queremos demostrar que \( A \) es Dedekind-finito si y sólo si \( B \) lo es.

    Supongamos que \( A \) es Dedekind-finito. Nuestro objetivo es demostrar que \( B \) también es Dedekind-finito. Por contradicción, consideremos que \( B \) no es Dedekind-finito. Esto implica que existe una función inyectiva \( \func{g}{B}{B} \) que no es sobreyectiva. 

    Tomemos $\func{\varphi}{A}{B}$ la función biyectiva que existe gracias a que $A\sim b$. Definimos 
    \[
        \func{f=\varphi^{-1} \circ g \circ \varphi}{A}{A}. 
    \]
    Dado que \( \varphi \) es biyectiva y $g$ es inyectiva, \( f \) es inyectiva. Además, como \( g \) no es sobreyectiva, \( f \) tampoco lo será. Esto implica que \( A \) es Dedekind-infinito, lo cual contradice nuestra suposición inicial.

\end{proof}
    
\end{document}