\documentclass[a4,10pt]{aleph-notas}

% -- Paquetes adicionales
\usepackage{enumitem}
\usepackage{aleph-comandos}

% -- Datos
\institucion{Proyecto Alephsub0}
\asignatura{Teoría de conjuntos}
\tema{Problema 435-1 del libro de Abhijit Dasgupta}
\autor{Andrés Merino}
\fecha{Diciembre 2024}

\fuente{montserrat}
\logouno[4.5cm]{Logos/LogoAlephsub0-02}
\definecolor{colordef}{cmyk}{0.81,0.62,0.00,0.22}

% -- Comandos adicionales
\newtheorem*{prob}{Problema}
    \tcolorboxenvironment{prob}{%
        color=colordef,recuadrost,colback=colordef!7,drop fuzzy shadow
    }


\begin{document}

\encabezado

\noindent
El siguiente problema corresponde al Problema 435-1 del libro Set theory de Abhijit Dasgupta. Primero, presentemos algunas definiciones:

% \begin{defi}[Potencia de Conjuntos]
%     Sea \(A, B\) conjuntos. La potencia \(A^B\) se define como el conjunto de todas las funciones de \(B\) en \(A\), es decir, 
%     \[
%         A^B = \{f : \func{f}{A}{B}\}
%     \]
% \end{defi}

\begin{prob}
    Sea $\mathcal{C}(\mathbb{R}) = \{ \func{f}{\mathbb{R}}{\mathbb{R}} : f\text{ es continua} \}$. Se tiene que
    \[
        \big|\mathcal{C}(\mathbb{R})\big| = \mathfrak{c}.
    \]
\end{prob}

\begin{proof}
    Para cada $x_0 \in \mathbb{R}$, consideremos la función constante $\func{f_{x_0}}{\mathbb{R}}{\mathbb{R}}$ dada por
    \[
        f_{x_0}(x) = x_0.
    \]
    Definimos la aplicación
    \[
        \funcion{\Phi}{\mathbb{R}}{\mathcal{C}(\mathbb{R})}{x_0}{\Phi(x_0) = f_{x_0}.}
    \]
    Esta función es inyectiva, pues si $\Phi(x_0) = \Phi(x_1)$, entonces $f_{x_0} = f_{x_1}$, lo que implica que $x_0 = x_1$. Así, se tiene que
    \[
        \big|\mathcal{C}(\mathbb{R})\big| \geq |\mathbb{R}| = \mathfrak{c}.
    \]

    Ahora, consideremos el conjunto $\mathcal{C}(\mathbb{Q}) = \{ \func{f}{\mathbb{Q}}{\mathbb{R}} : f \text{ es continua} \}$. Definimos la función
    \[
        \funcion{\Psi}{\mathcal{C}(\mathbb{R})}{\mathcal{C}(\mathbb{Q})}{f}{\Psi(f) = f|_{\mathbb{Q}},}
    \]
    donde $\Psi(f)$ es la restricción de $f$ a $\mathbb{Q}$. 

    La función $\Psi$ es inyectiva, ya que si $\Psi(f) = \Psi(g)$, entonces $f$ y $g$ coinciden en todos los puntos racionales. Como los racionales son densos en $\mathbb{R}$ y $f, g$ son continuas en $\mathbb{R}$, se sigue que $f = g$ en todo $\mathbb{R}$. 

    Para ver que $\Psi$ es sobreyectiva, tomemos cualquier función $h \in \mathcal{C}(\mathbb{Q})$. Dado que $h$ es continua en $\mathbb{Q}$, se puede extender a una función continua en $\mathbb{R}$ (por el teorema de extensión de continuidades en espacios métricos). Así, existe una función $F \in \mathcal{C}(\mathbb{R})$ tal que $F|_{\mathbb{Q}} = h$, lo que prueba que $\Psi$ es sobreyectiva.

    Como
    \[
        \big|\mathcal{C}(\mathbb{Q})\big| = \big|\mathbb{R}^{\mathbb{Q}}\big| = \big|\mathbb{R}^{\mathbb{N}}\big| = \mathfrak{c}^{\aleph_0} = \mathfrak{c},
    \]
    se sigue que
    \[
        \big|\mathcal{C}(\mathbb{R})\big| \leq \mathfrak{c}.
    \]

    Con esto, concluimos que
    \[
        \big|\mathcal{C}(\mathbb{R})\big| = \mathfrak{c}.\qedhere
    \]
\end{proof}


    
\end{document}