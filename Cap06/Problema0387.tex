\documentclass[a4,10pt]{aleph-notas}

% -- Paquetes adicionales
\usepackage{enumitem}
\usepackage{aleph-comandos}

% -- Datos
\institucion{Proyecto Alephsub0}
\asignatura{Teoría de conjuntos}
\tema{Problema 387 del libro de Abhijit Dasgupta}
\autor{Andrés Merino}
\fecha{Diciembre 2024}

\fuente{montserrat}
\logouno[4.5cm]{Logos/LogoAlephsub0-02}
\definecolor{colordef}{cmyk}{0.81,0.62,0.00,0.22}

% -- Comandos adicionales
\newtheorem*{prob}{Problema}
    \tcolorboxenvironment{prob}{%
        color=colordef,recuadrost,colback=colordef!7,drop fuzzy shadow
    }


\begin{document}

\encabezado

\noindent
El siguiente problema corresponde al Problema 387 del libro Set theory de Abhijit Dasgupta. Primero, presentemos algunas definiciones:

\begin{prob}
    Se tiene que
        \[\sum_{n\in\mathbb{N}^*} n = 1 + 2 + 3 +\cdots =  \aleph_0.\]
\end{prob}

\begin{proof}[Demostración usando conjuntos]
    Sea $(A_n)_{n\in\mathbb{N}^*}$ una familia de conjuntos, disjuntos dos a dos, tal que \(|A_n|=n\), para todos $n\in\mathbb{N}^*$. Tomemos
    \[
        A=\bigcup_{n\in\mathbb{N}^*} A_n,
    \]
    por lo tanto
    \[
        \sum_{n\in\mathbb{N}^*} n 
        = \left|\bigcup_{n\in\mathbb{N}^*} A_n\right|
        =|A|.
    \]
    Como cada \(A_n\) es finito, se tiene que $A$ es la unión numerable de conjuntos finitos, por lo tanto, es, a lo más, numerable, es decir, \(|A|\le \aleph_0\). 
    
    Además, dado que cada \(A_n\) es no vacío, por el axioma de elección, existe una función
    \[
        \func{f}{\mathbb{N}^*}{\bigcup_{n\in\mathbb{N}^*} A_n}
    \]
    tal que $f(n)\in A_n$, para todo $n\in \mathbb{N}^*$. Como $(A_n)_{n\in\mathbb{N}^*}$ son disjuntos dos a dos, esta función es inyectiva, lo que implica \(\aleph_0\le |A|\). 
    
    Por tanto,
    \(
        |A|=\aleph_0,
    \)
    es decir,
    \[
        \sum_{n\in\mathbb{N}^*} n = \aleph_0.
    \]
\end{proof}


\begin{proof}[Demostración usando aritmética cardinal]
    Por un lado,
    \[
        \sum_{n\in\mathbb{N}^*} n 
        \geq \sum_{n\in\mathbb{N}^*} 1
        = 1\cdot |\mathbb{N}^*|
        = 1\cdot \aleph_0
        = \aleph_0.
    \]
    Por otro lado,
    \[
        \sum_{n\in\mathbb{N}^*} n 
        \leq \sum_{n\in\mathbb{N}^*} \aleph_0
        = \aleph_0\cdot |\mathbb{N}^*|
        = \aleph_0\cdot \aleph_0
        = \aleph_0^2
        = \aleph_0.
    \]
    Por lo tanto,
    \[
        \sum_{n\in\mathbb{N}^*} n = \aleph_0.
    \]
\end{proof}

\begin{prob}
    Se tiene que
        \[\prod_{n\in\mathbb{N}^*} n = 1 \cdot 2 \cdot 3 \cdot\cdots =  2^{\aleph_0.}\]
\end{prob}

\begin{proof}[Demostración usando aritmética cardinal]
    Por un lado,
    \[
        \prod_{n\in\mathbb{N}^*} n 
        = 1\cdot\prod_{n\in\mathbb{N}^*\setminus \{1\}} n 
        \geq \prod_{n\in\mathbb{N}^*\setminus \{1\}} 2 
        = 2^{|\mathbb{N}^*\setminus \{1\}|}
        = 2^{\aleph_0}.
    \]
    Por otro lado,
    \[
        \prod_{n\in\mathbb{N}^*} n 
        \leq \prod_{n\in\mathbb{N}^*} 2^{\aleph_0 }
        = \left(2^{\aleph_0} \right)^{|\mathbb{N}^*|} 
        = \left(2^{\aleph_0} \right)^{\aleph_0} 
        = 2^{\aleph_0\cdot \aleph_0}
        = 2^{\aleph_0^2}
        = 2^{\aleph_0}.
    \]
    Por lo tanto,
    \[
        \prod_{n\in\mathbb{N}^*} n = 2^{\aleph_0}.
    \]
\end{proof}
\end{document}