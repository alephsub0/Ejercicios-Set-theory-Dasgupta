\documentclass[a4,10pt]{aleph-notas}

% -- Paquetes adicionales
\usepackage{enumitem}
\usepackage{aleph-comandos}

% -- Datos
\institucion{Proyecto Alephsub0}
\asignatura{Teoría de conjuntos}
\tema{Problema 368 del libro de Abhijit Dasgupta}
\autor{Andrés Merino}
\fecha{Diciembre 2024}

\fuente{montserrat}
\logouno[4.5cm]{Logos/LogoAlephsub0-02}
\definecolor{colordef}{cmyk}{0.81,0.62,0.00,0.22}

% -- Comandos adicionales
\newtheorem*{prob}{Problema}
    \tcolorboxenvironment{prob}{%
        color=colordef,recuadrost,colback=colordef!7,drop fuzzy shadow
    }


\begin{document}

\encabezado

\noindent
El siguiente problema corresponde al Problema 368 del libro Set theory de Abhijit Dasgupta. Primero, presentemos algunas definiciones:

\begin{defi}[Potencia de Conjuntos]
    Sea \(A, B\) conjuntos. La potencia \(A^B\) se define como el conjunto de todas las funciones de \(B\) en \(A\), es decir, 
    \[
        A^B = \{f : \func{f}{A}{B}\}
    \]
\end{defi}

\begin{prob}
    Sean $A, B$ conjuntos. Si \(A \sim C\) y \(B \sim D\), entonces \(A^B \sim C^D\).
\end{prob}

\begin{proof}
    Sean \(\func{f}{A}{C}\) y \(\func{g}{B}{D}\) biyectivas.  
    Para cada \(F \in A^B\) (es decir, \(\func{F}{B}{A}\)), podemos construir la función:
    \[
        f \circ F \circ g^{-1}: D \to C.
    \]
    Así, definimos una función:
    \[
        \funcion{\mathcal{F}}{A^B}{C^D}{F}{f \circ F \circ g^{-1}}.
    \]
    Probemos que \(\mathcal{F}\) es biyectiva:
    \begin{enumerate}
        \item \textbf{Inyectividad:}  
        Sean \(F, G \in A^B\) tales que \(\mathcal{F}(F) = \mathcal{F}(G)\). Entonces:
        \[
        f \circ F \circ g^{-1} = f \circ G \circ g^{-1}.
        \]
        Como \(f\) y \(g\) son biyectivas, se concluye que \(F = G\). Por lo tanto, \(\mathcal{F}\) es inyectiva.
        
        \item \textbf{Sobreyectividad:}  
        Sea \(H \in C^D\). Queremos encontrar \(F \in A^B\) tal que \(\mathcal{F}(F) = H\).  
        Definimos \(\func{F=f^{-1} \circ H \circ g}{B}{A}\), con esto:
        \[
            \mathcal{F}(F) = f \circ (f^{-1} \circ H \circ g) \circ g^{-1} = H.
        \]
        Por lo tanto, \(\mathcal{F}\) es sobreyectiva.
    \end{enumerate}
    
    Dado que \(\mathcal{F}\) es inyectiva y sobreyectiva, concluimos que es biyectiva. Por lo tanto, \(A^B \sim C^D\).

\end{proof}

    
\end{document}