\documentclass[a4,10pt]{aleph-notas}

% -- Paquetes adicionales
\usepackage{enumitem}
\usepackage{aleph-comandos}

% -- Datos
\institucion{Proyecto Alephsub0}
\asignatura{Teoría de conjuntos}
\tema{Problema 435-1 del libro de Abhijit Dasgupta}
\autor{Andrés Merino}
\fecha{Diciembre 2024}

\fuente{montserrat}
\logouno[4.5cm]{Logos/LogoAlephsub0-02}
\definecolor{colordef}{cmyk}{0.81,0.62,0.00,0.22}

% -- Comandos adicionales
\newtheorem*{prob}{Problema}
    \tcolorboxenvironment{prob}{%
        color=colordef,recuadrost,colback=colordef!7,drop fuzzy shadow
    }


\begin{document}

\encabezado

\noindent
El siguiente problema corresponde al Problema 435-1 del libro Set theory de Abhijit Dasgupta. Primero, presentemos algunas definiciones:

% \begin{defi}[Potencia de Conjuntos]
%     Sea \(A, B\) conjuntos. La potencia \(A^B\) se define como el conjunto de todas las funciones de \(B\) en \(A\), es decir, 
%     \[
%         A^B = \{f : \func{f}{A}{B}\}
%     \]
% \end{defi}

\begin{prob}
    Sea $\mathcal{R}[0,1] = \{ \func{f}{[0,1]}{\mathbb{R}} : f\text{ es acotada y Riemann integrable} \}$. Se tiene que 
    \[
        \big|\mathcal{R}[0,1]\big| = 2^{\mathfrak{c}} = \mathfrak{f}.
    \]
\end{prob}

\begin{proof}
    Como $\mathcal{R}[0,1] \subseteq \mathbb{R}^{[0,1]}$, entonces
    \[
        \big|\mathcal{R}[0,1]\big| 
        \leq \big|\mathbb{R}^{[0,1]}\big| 
        = \big|\mathbb{R}^{\mathbb{R}}\big| 
        = \mathfrak{c}^{\mathfrak{c}} = \mathfrak{f}.
    \]
    
    Por otro lado, sea $C$ el conjunto de Cantor. Para cada $f \in \mathcal{R}[0,1]$, es decir, para cada función $\func{f}{C}{[0,1]}$, podemos definir 
    \[
        \funcion{\hat{f}}{[0,1]}{[0,1]}{x}{
        \hat{f}(x) =
        \begin{cases}
            f(x) & \text{si } x \in C, \\
            0 & \text{si } x \notin C.
        \end{cases}}
    \]
    Esta función es acotada y constante en el complemento de un  conjunto de medida cero, por lo tanto, es Riemann-integrable. Así, podemos definir la función inyectiva 
    \[
        \funcion{\mathcal{F}}{[0,1]^C}{\mathcal{R}[0,1]}{f}{\mathcal{F}(f)=\hat{f}.}
    \]
    
    Para ver que es inyectiva, tomemos $f, g \in [0,1]^C$ tales que $\mathcal{F}(f)=\mathcal{F}(g)$, es decir, $\hat{f} = \hat{g}$. Entonces, para todo $x \in [0,1]$,
    \[
        \hat{f}(x) = \hat{g}(x),
    \]
    en particular, para todo $x \in C$ se tiene 
    \[
        f(x) = \hat{f}(x) = \hat{g}(x) = g(x).
    \]
    Por lo tanto, $f = g$. Como $\hat{f}$ es inyectiva, se sigue que
    \[
        \big|\mathcal{R}[0,1]\big| \geq \big|[0,1]^C\big| = \big|\mathbb{R}^{\mathbb{R}}\big| = \mathfrak{c}^\mathfrak{c}=\mathfrak{f}.
    \]
    
    De esta forma, concluimos que 
    \[
        \big|\mathcal{R}[0,1]\big| = \mathfrak{f}.\qedhere
    \]
\end{proof}

    
\end{document}