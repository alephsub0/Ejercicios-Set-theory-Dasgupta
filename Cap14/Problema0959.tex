\documentclass[a4,10pt]{aleph-notas}

% -- Paquetes adicionales
\usepackage{enumitem}
\usepackage{aleph-comandos}

% -- Datos
\institucion{Proyecto Alephsub0}
\asignatura{Teoría de conjuntos}
\tema{Ejercicios 959 del libro de Abhijit Dasgupta}
\autor{Andrés Merino}
\fecha{Octubre 2024}

\fuente{montserrat}
\logouno[4.5cm]{Logos/LogoAlephsub0-02}
\definecolor{colordef}{cmyk}{0.81,0.62,0.00,0.22}

% -- Comandos adicionales
\newtheorem*{prob}{Problema}
    \tcolorboxenvironment{prob}{%
        color=colordef,recuadrost,colback=colordef!7,drop fuzzy shadow
    }


\begin{document}

\encabezado

%%%%%%%%%%%%%%%%%%%%%%%%%%%%%%%%%%%%%%%%
\section{Conjuntos Reales y Funciones}
%%%%%%%%%%%%%%%%%%%%%%%%%%%%%%%%%%%%%%%%

\noindent
El siguiente problema corresponde al Problema 959 del libro Set theory de Abhijit Dasgupta.

\begin{prob}[959]
    Un conjunto $A$ es denso en ninguna parte si y solo si todo intervalo abierto no vacío contiene un subintervalo abierto no vacío que es disjunto con $A$.
\end{prob}

\begin{ejer}[959]
    Un conjunto $A$ es denso en ninguna parte si y solo si todo intervalo abierto no vacío contiene un subintervalo abierto no vacío que es disjunto con $A$.
\end{ejer}

\begin{proof}
    Se deja al lector.
\end{proof}
    
\end{document}